\documentclass{report}
\usepackage[layout=letterpaper,margin=1in]{geometry}

\title{Interactive Content Specification}
\author{Satvik Borra}
\date{\today}

\usepackage[endianness=big]{bytefield}
\bytefieldsetup{boxformatting={\centering\footnotesize}}

\usepackage[utf8]{inputenc}
\usepackage[english]{babel}
\usepackage{xcolor}
\usepackage{graphics}
\usepackage{amsmath}

\pagecolor{white}
\begin{document}

\maketitle

\tableofcontents

\part{Introduction}

\chapter{Introduction}

\section{Purpose}
The Interactive Content (IC) File Format is a file format for modern day uses. 
IC files allows the user to interact with the content presented in robust ways.
The format offers a way to layout content in a flexible way and is able to present many forms of content.

\section{Uses}
IC allows for creativity in showing a variety of content. For example, entire newspapers, dictionaries, text-based games, and ... can all be displayed using IC. All of the interactive features are also available when displaying most content types. See the EXAMPLES section for in depth use cases.

\part{Specification}

\chapter{Overview}
\section{Terminology}
A chunk refers to all the bytes corresponding to its ID and its data.
An element refers to a chunk with type Container or Content.

\section{Bytes Summary}
The IC format uses big-endian for reading and writing bytes. Bytes are separated by chunks.
\section{The ID System}
Each element has a chunk identifier and an ID number associated with it. All IDs must be unique within a chunk type. ID Components are used within Chunks to refer to other Chunks.
\subsection{The ID Component}
For each chunk, two bytes (four bytes if the EXTENDED RANGE is used) at the start of the chunk definition are used for its unique identification. \\
The three left-most bits define the chunk type. The thirteen right-most bits define the chunk ID.
\definecolor{lightgreen}{rgb}{0.64,1,0.71}
\definecolor{lightred}{rgb}{1,0.7,0.71}
\begin{center}
\begin{bytefield}[bitwidth=2em,bitheight=\widthof{~Sign~}]{16} \\
\bitheader{0-15} \\
\bitbox{3}[bgcolor=lightgreen]{Chunk Type} & \bitbox{13}[bgcolor=lightred]{Chunk ID}
\end{bytefield}
\end{center}
\subsection{Chunk Type Identifiers}
Each Chunk type has its own unique identifier that appears as the first 3 bits in the chunk. \\
The following bits represent each type of chunk:
\begin{itemize}
\item 000 - reserved, special identifier to end element list within containers
\item 001 - A Container chunk.
\item 010 - A Content chunk.
\item 011 - A Layout chunk.
\item 100 - A Action chunk.
\item 101 - A Styling chunk.
\item 110 - A Header chunk.
\item 111 - extended range, see EXTENDED IDs
\end{itemize}

Chunk types in extended range, which start with 111 to signify using the extended range:
\begin{itemize}
	\item 111 01000 - A future chunk with type 8.
\end{itemize}
\subsection{ID Numbers}
All chunks must have a unique ID. IDs do not have to be unique between chunk types. An ID is 13 bits if not using the extended range, and 24 bits otherwise.

\subsection{Extended IDs}
\paragraph{}
To allow for more than 5 chunk types and 8,192 ID numbers per chunk type, an extended bit range can be used. Using the extended range, 5 bits are used for the chunk type and 24 bits for the ID. Chunk types 1 to 5 can still be used in the extended range to allow for more ID numbers of those chunk types (containers, content, layouts, actions, and styling).
\paragraph{}
To use the extended range, the chunk type is prefixed with 111. By using this prefix, it signifies the next 5 bits (until the end of the first full byte) is used for the chunk type.

\definecolor{lightcyan}{rgb}{0.84,1,1}
\definecolor{lightgreen}{rgb}{0.64,1,0.71}
\definecolor{lightred}{rgb}{1,0.7,0.71}
\begin{center}
\begin{bytefield}[bitwidth=1.4em,bitheight=\widthof{~Sign~}]{32}
\bitheader{31,28,23,0} \\
\bitbox{3}[bgcolor=lightcyan]{\rotatebox{0}{111}}
\bitbox{5}[bgcolor=lightgreen]{Chunk Type}
\bitbox{24}[bgcolor=lightred]{Chunk ID}
\end{bytefield}
\end{center}


\subsection{ID Limitations}

For a standard ID, two bytes are used. The first 3 bits are the chunk type, and the next 13 bits is the chunk ID. This results in $2^{3}$ = 8 types - 2 reserved types - 1 extended range type = 5 valid chunk types. After the chunk type, the next 13 bits are the ID of the element. This allows for $2^{13}$ = 8,192 unique IDs per chunk type.

If using the extended range, 5 bits are used for the chunk type, with the next 24 bits for the chunk ID. With 5 bits for the chunk type, $2^{5}$ = 32 new chunk types are available. 32 new types - 5 existing types (not in the extended range) - 3 reserved types = 24 new chunk types. 

There are three bytes (24 bits) used to define the chunk ID when using the extended range. This results in a max number of $2^{24}$ = 16,777,216 IDs.\\
\section{The Positioning System}
The position and size of Elements and Actions are defined by rectangles which defines its size based on a proportion of its parent element.
\subsection{Percentage Rectangles}
A complete rectangle is defined with 4 bytes. Each byte is unsigned, meaning the range of values is 0-255.
The four bytes define a rectangle: X, Y, Width, and Height.\\
\begin{center}
\begin{bytefield}[bitwidth=1.4em,bitheight=\widthof{~Sign~}]{32}
\bitheader{0,7,15,23,31} \\
\bitbox{8}{X} & \bitbox{8}{Y} &
\bitbox{8}{Width} & \bitbox{8}{Height}
\end{bytefield}
\end{center}
This value represents the percentage of its parent element. This means usually the byte values will be between 0x0 and 0x32 (100 in decimal, representing 100\% of the parent).
The X and Y percents are multiplied by the parent element's width and height, respectively.

\section{Magic Number}
The magic number for IC files are the first bytes in the file.
The magic number for IC files is 73 61 6d 61 6c 69 74 6c 6e 76 75 61

\section{Specification Versioning}
The version of the Interactive Content specification is included in every file in the HEADER chunk. Specification Versions follow SimVer versioning (MAJOR.MINOR.PATCH). The specification version in an IC file is 4 bytes. The first two bytes define the MAJOR version. The next byte defines the MINOR version. The last byte defines the PATCH version. This means version 1.0.0 of the Interactive Content Specification is defined in an IC file as would be 0x0001 00 00.
\section{Structure}
All IC files start with the magic number. Then the next chunk is a HEADER chunk.
\chapter{Chunks}

\section{Header}
The header chunk must be at the top of a IC file after the MAGIC NUMBER. 
Headers have chunk type 8.
This means the extended chunk range must be used.

The chunk type for headers is 11001000. 

\subsection{Static Properties}
Static header properties are in every header file and are defined to be in the same static location from the ID. 

\subsubsection{File Version}
The file version specifies the version of the IC specification the rest of the file follows. The file version is four bytes.\\
The current file version is 1.0.0 (0x0001 00 00).

\subsubsection{Starting Container ID}
The starting container ID defines the the container that is shown first. The starting container is the starting point of the IC file. It is a ID COMPONENT, so either 2 bytes or 4 bytes are used to identify the starting container. The chunk type must be a container.

\subsection{Dynamic Properties}
Headers can have any number of dynamic properties and works as a list of key-value pairs. Strings must be null terminated before the next string.
\subsubsection{Metadata}
Metadata is any key in dynamic properties that is not a Recognized Setting. Metadata is simply shown in the about menu when opening a file.
\subsubsection{File Settings}
File settings are predefined keys which change the behavior of the file. 
\subparagraph{Recognized Settings}
The following keys are used:\\
\\
"win\_aspect" defines the aspect ratio of the window that should be kept when the window is opened and resized.\\
"win\_rect" defines the starting window size when opening the file. 4 bytes in percent of screen size. Refer to RECTANGLE POSITIONING for more information.

The following keys are used to manage automatically updating the file:\\
file\_version defines the current version of the file. The value is a unsigned 2 byte number.\\
version\_url defines the URL to retrieve the latest version of the current file from. The remote file must be a length of two bytes, defining the latest version of the file in hex.\\
update\_url defines the URL where the updated version of the entire file is located.


\subsection{Automatically Updating the File}
If the file\_version value is higher than the value retrieved from the version\_url key, the update\_url value is used to download and replace the currently opened file. If the value from the version\_url key is 0xFFFF, the update\_url is always used.

\subsection{Bytes}
The first bytes of a Header is an ID COMPONENT. Because the Header is an extended chunk type (chunk type > 7), the 4 bytes after the ID COMPONENT is the length of the chunk.

\section{Container}
Containers are the backbone of an IC file. Containers are responsible for displaying content and define which LAYOUT to follow. All CONTENT must be a part of a Container for it to be displayed. A container can contain other containers and/or content.\\

\subsection{Bytes}
The first bytes of the Container is an ID COMPONENT.

\section{Layout}
A layout defines the positions of elements. 
\section{Style}
\subsection{Style Properties}
\subsubsection{Window Styling}
\subsubsection{Text Styling}
\section{Content}
\subsection{Actions in Content}
\subsection{Content Styling}
\subsubsection{Global Styling}
Only an overview of the global style.
\subsubsection{Individual Styling}
Styles are handled individually by the content. These styles are only supported by the following content types:
\begin{itemize}
\item Text
\item Button
\item Bitmap
\end{itemize}
\subsubsection{Styling Order}

\section{Extended Chunk Types}
\subsection{Maintaining Compatibility}
To maintain compatibility with previous IC specification versions, all chunk types in the extended range must include the length of the chunk in bytes after defining its ID COMPONENT. This way, if an unknown chunk type is encountered, it can be skipped.
\subsection{Suggesting a new Chunk Type}
To suggest a new chunk type, follow this process:

\chapter{Content}
\section{Text}
\subsection{Byte Structure}
\subsection{Options}
Styles can be applied to text by the following byte sequence: 0xEEBC80, and ended with 0xEEBC81.\\
Actions can be started with 0xEEBC82 and ended with 0xEEBC83.
\subsection{Styling}
\section{Button}
\subsection{Byte Structure}
\subsection{Styling}
\section{Image}
\subsection{Byte Structure}
\subsection{Options}
Key value pairs
\section{Bitmap}
\subsection{Byte Structure}
\subsection{Styling}
For each shape type: line, rectangle, and circle.
\section{Web Page}
\subsection{Byte Structure}
\section{Suggesting a new Content Type}
To suggest a new content type, follow this process:

\end{document}