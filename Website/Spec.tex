\documentclass{report}
\usepackage[layout=letterpaper,margin=1in]{geometry}

\title{Interactive Content Specification}
\author{Satvik Borra}
\date{\today}

\usepackage[endianness=big]{bytefield}
\bytefieldsetup{boxformatting={\centering\footnotesize}}

\usepackage[utf8]{inputenc}
\usepackage[english]{babel}
\usepackage{xcolor}
\usepackage{graphics}
\usepackage{amsmath}
\usepackage{tabularx}
\usepackage{makecell}

\pagecolor{white}
\begin{document}

\maketitle

\tableofcontents

\part{Introduction}

\chapter{Introduction}

\section{Purpose}
The Interactive Content (IC) File Format is a file format for modern day uses. 
IC files allows the user to interact with the content presented in robust ways.
The format offers a way to layout content in a flexible way and is able to present many forms of content.

\section{Uses}
IC allows for creativity in showing a variety of content. For example, entire newspapers, dictionaries, text-based games, and ... can all be displayed using IC. All of the interactive features are also available when displaying most content types. See the EXAMPLES section for in depth use cases.

\part{Specification}

\chapter{Overview}
\section{Terminology}
A chunk refers to all the bytes corresponding to its ID and its data.
An element refers to a chunk with type Container or Content.

\section{Bytes Summary}
The IC format uses big-endian for reading and writing bytes. Bytes are separated by chunks.
\section{The ID System}
Each element has a chunk identifier and an ID number associated with it. All IDs must be unique within a chunk type. ID Components are used within Chunks to refer to other Chunks.
\subsection{The ID Component}
For each chunk, two bytes (four bytes if the EXTENDED RANGE is used) at the start of the chunk definition are used for its unique identification. \\
The three left-most bits define the chunk type. The thirteen right-most bits define the chunk ID.
\definecolor{lightgreen}{rgb}{0.64,1,0.71}
\definecolor{lightred}{rgb}{1,0.7,0.71}
\begin{center}
\begin{bytefield}[bitwidth=2em,bitheight=\widthof{~Sign~}]{16} \\
\bitheader{0-15} \\
\bitbox{3}[bgcolor=lightgreen]{Chunk Type} & \bitbox{13}[bgcolor=lightred]{Chunk ID}
\end{bytefield}
\end{center}
\subsection{Chunk Type Identifiers}
Each Chunk type has its own unique identifier that appears as the first 3 bits in the chunk. \\
The following bits represent each type of chunk:
\begin{itemize}
\item 000 - reserved, special identifier to end element list within containers
\item 001 - A Container chunk.
\item 010 - A Content chunk.
\item 011 - A Layout chunk.
\item 100 - A Action chunk.
\item 101 - A Styling chunk.
\item 110 - A Header chunk.
\item 111 - extended range, see EXTENDED IDs
\end{itemize}

Chunk types in extended range, which start with 111 to signify using the extended range:
\begin{itemize}
	\item 111 01000 - A future chunk with type 8.
\end{itemize}
\subsection{ID Numbers}
All chunks must have a unique ID. IDs do not have to be unique between chunk types. An ID is 13 bits if not using the extended range, and 24 bits otherwise.

\subsection{Extended IDs}
\paragraph{}
To allow for more than 5 chunk types and 8,192 ID numbers per chunk type, an extended bit range can be used. Using the extended range, 5 bits are used for the chunk type and 24 bits for the ID. Chunk types 1 to 5 can still be used in the extended range to allow for more ID numbers of those chunk types (containers, content, layouts, actions, and styling).
\paragraph{}
To use the extended range, the chunk type is prefixed with 111. By using this prefix, it signifies the next 5 bits (until the end of the first full byte) is used for the chunk type.

\definecolor{lightcyan}{rgb}{0.84,1,1}
\definecolor{lightgreen}{rgb}{0.64,1,0.71}
\definecolor{lightred}{rgb}{1,0.7,0.71}
\begin{center}
\begin{bytefield}[bitwidth=1.4em,bitheight=\widthof{~Sign~}]{32}
\bitheader{31,28,23,0} \\
\bitbox{3}[bgcolor=lightcyan]{\rotatebox{0}{111}}
\bitbox{5}[bgcolor=lightgreen]{Chunk Type}
\bitbox{24}[bgcolor=lightred]{Chunk ID}
\end{bytefield}
\end{center}


\subsection{ID Limitations}

For a standard ID, two bytes are used. The first 3 bits are the chunk type, and the next 13 bits is the chunk ID. This results in $2^{3}$ = 8 types - 2 reserved types - 1 extended range type = 5 valid chunk types. After the chunk type, the next 13 bits are the ID of the element. This allows for $2^{13}$ = 8,192 unique IDs per chunk type.

If using the extended range, 5 bits are used for the chunk type, with the next 24 bits for the chunk ID. With 5 bits for the chunk type, $2^{5}$ = 32 new chunk types are available. 32 new types - 5 existing types (not in the extended range) - 3 reserved types = 24 new chunk types. 

There are three bytes (24 bits) used to define the chunk ID when using the extended range. This results in a max number of $2^{24}$ = 16,777,216 IDs.\\
\section{The Positioning System}
The position and size of Elements and Actions are defined by rectangles which defines its size based on a proportion of its parent element.
\subsection{Percentage Rectangles}
A complete rectangle is defined with 4 bytes. Each byte is unsigned, meaning the range of values is 0-255.
The four bytes define a rectangle: X, Y, Width, and Height.\\
\begin{center}
\begin{bytefield}[bitwidth=1.4em,bitheight=\widthof{~Sign~}]{32}
\bitheader{0,7,15,23,31} \\
\bitbox{8}{X} & \bitbox{8}{Y} &
\bitbox{8}{Width} & \bitbox{8}{Height}
\end{bytefield}
\end{center}
This value represents the percentage of its parent element. This means usually the byte values will be between 0x0 and 0x32 (100 in decimal, representing 100\% of the parent).
The X and Y percents are multiplied by the parent element's width and height, respectively.

\section{Magic Number}
The magic number for IC files are the first bytes in the file.
The magic number for IC files is 73 61 6d 61 6c 69 74 6c 6e 76 75 61

\section{Specification and File Versioning}
The version of the Interactive Content specification is included in every file in the HEADER chunk. Specification Versions follow SimVer versioning (MAJOR.MINOR.PATCH). The specification version, and optional file version, in an IC file is 4 bytes. The first two bytes define the MAJOR version. The next byte defines the MINOR version. The last byte defines the PATCH version. This means version 1.0.0 of the Interactive Content Specification is defined in an IC file as would be 0x0001 00 00.
\section{Structure}
All IC files start with the magic number. Then the next chunk is a HEADER chunk.
\chapter{Chunks}

\section{Header}
The header chunk must be at the top of a IC file after the MAGIC NUMBER. 
Headers have chunk type 7.
The chunk type for headers is 110. 

\subsection{Static Properties}
Static header properties are in every header file and are defined to be in the same static location from the ID. 

\subsubsection{File Version}
The file version specifies the version of the IC specification the rest of the file follows. The file version is four bytes.\\
The current file version is 1.0.0 (0x0001 00 00).

\subsubsection{Initial Container ID}
The initial container ID defines the the container that is shown when the file is first opened, and is the starting point of the IC file. It is a ID COMPONENT, so either 2 bytes or 4 bytes are used to identify the starting container. The chunk type must be a container.

\subsubsection{Initial Container Style ID}
The initial container STYLE ID is a ID COMPONENT that defines the container wide style used for the starting parent container.

\subsection{Dynamic Properties}
Headers can have any number of dynamic properties and works as a list of key-value pairs. Strings must be null terminated before the next string.
\subsubsection{Metadata}
Metadata is any key in dynamic properties that is not a Recognized Setting. Metadata is simply shown in the about menu when opening a file.
\subsubsection{File Settings}
File settings are predefined keys which change the behavior of the file. 
\subparagraph{Recognized Settings}
The following keys are used:\\
\begin{tabularx}{\textwidth}{l|l|X}
  \textbf{\makecell[cl]{Setting Key}} & \textbf{Value Format} & \textbf{Description}\\
\hline
"win\_aspect" & XX XX & Defines the aspect ratio of the window when the file is opened and the window is resized. The first byte defines the width, the second byte defines the height. For example, to keep the window at a 16:9 aspect ratio, use the bytes 10 09.\\
"win\_rect" & XX XX XX XX & Defines the starting window size when opening the file. Refer to RECTANGLE POSITIONING for more information.\\
\\
%\end{tabularx}
%The following keys are used to manage automatically updating the file:\\
%\begin{tabularx}{\textwidth}{l|l|X}

"file\_version" & XX XX XX XX & Defines the current version of the file (different from the specification version). See SPECIFICATION AND FILE VERSIONING for more information.\\

"version\_url" & X*N & Defines the URL to retrieve the latest version of the current file from. The remote file must be a length of two bytes, defining the latest version of the file in hex.\\
"update\_url" & X*N & Defines the URL where the updated version of the entire file is located.

\end{tabularx}

\subsection{Automatically Updating the File}
If the file\_version value is higher than the value retrieved from the version\_url key, the update\_url value is used to download and replace the currently opened file. If the value from the version\_url key is 0xFFFF, the update\_url is always used.

\subsection{Bytes}
All Headers must be at least 9 bytes. The first bytes of a Header is an ID COMPONENT. Next is the static properties: 4 bytes for the SPECIFICATION VERSION and an ID COMPONENT to define the starting Container. 

\begin{center}
\begin{bytefield}{32} \\
\bitheader{0,7,15,23,31} \\
\begin{rightwordgroup}{Static Properties}
\bitbox{16}{Header ID} & \bitbox{16}{Extended ID} \\
\bitbox{32}{Specification Version} \\
\bitbox{16}{Initial Container ID} & \bitbox{16}{Extended ID} \\
\bitbox{16}{Initial Container Layout ID} & \bitbox{16}{Extended ID} 
\end{rightwordgroup}\\
\begin{rightwordgroup}{Key Value Pairs}\\
\bitbox{16}{Key 1 (null terminated ASCII string)} & \bitbox{16}{Value 1} \\
\wordbox[]{1}{$\vdots$} \\[1ex]
\bitbox{16}{Key N (null terminated ASCII string)} & \bitbox{16}{Value N} 
\end{rightwordgroup} \\

\bitbox{8}{End Header Byte}
\end{bytefield}
\end{center}

\section{Container}
Containers are the backbone of an IC file. Containers are responsible for displaying content and define which LAYOUT to follow. All CONTENT must be a part of a Container for it to be displayed. A container can contain other containers and/or content.\\

\subsection{Bytes}
The first bytes of the Container is an ID COMPONENT.

\section{Layout}
A layout defines the number of elements and the positions of those elements. 
\subsection{Bytes}
\begin{center}
\begin{bytefield}{32} \\
\bitheader{0,7,15,23,31} \\
\bitbox{16}{Layout ID} & \bitbox{16}{Extended ID} \\
\bitbox{8}{Options Byte} & \bitbox{8}{\# Elements} \\
\begin{rightwordgroup}{Element N Position}\\
\bitbox{8}{Position X} & \bitbox{8}{Position Y} & \bitbox{8}{Width} & \bitbox{8}{Height} \\
\bitbox{16}{Style ID} & \bitbox{16}{Extended ID} \\
\end{rightwordgroup} 
\end{bytefield}
\end{center}


Options byte:
\begin{center}
\begin{bytefield}[bitwidth=0.07\linewidth]{8} \\
\bitheader{0-7} \\
\bitbox{1}{NA} & \bitbox{1}{NA} & \bitbox{1}{NA} & \bitbox{1}{NA} & \bitbox{1}{NA} & \bitbox{1}{NA} & \bitbox{1}{NA} & \bitbox{1}{Larger size}
\end{bytefield}
\end{center}
\section{Style}

\subsection{Style Keys}
The key to define a style property (the Byte Key) is one byte. If the first four bits of the style property if 1111 (0xFX) then an extra byte is read for the style key. \\
To improve compatibility across specification versions, it is recommended to write style keys introduced in recent versions later in the style. Once an unknown Byte Key is read, the rest of the style is skipped.
\subsection{Style Properties}
The following tables define the currently used Byte Keys and the values associated with each key. The Byte Key defines the two hex values used to make up the byte. The byte format shows the bytes following the Byte Key to determine the value. An X refers to any hex value, and an N refers to any number of bytes.
\subsubsection{Control Keys}
\begin{tabularx}{\textwidth}{l|l|l|X}
  \textbf{\makecell[cl]{Byte\\Key}} & \textbf{Key Name} & \textbf{Value Format} & \textbf{Description}\\
\hline
00 & End Style & -- & Once this key is read, it signifies the end of the style.\\
FX & \makecell[tl]{Using a style key\\with an extra byte} & XX (extra byte) & Use an extra byte to define the style key to allow for more keys in the future.\\
\end{tabularx}
\subsubsection{Window Styling}
\begin{tabularx}{\textwidth}{l|l|l|X}
  \textbf{\makecell[cl]{Byte\\Key}} & \textbf{Key Name} & \textbf{Value Format} & \textbf{Description}\\
\hline
01 & Border Width & XX XXXX & TODO The border width. The first byte determines how the width is determined (pixels, percent of width/height, etc). The next two bytes define the value.\\
02 & Border Color & XX XX XX & RGB value for border color\\
03 & Background Color & XX XX XX & RGB value for background color\\
\end{tabularx}
\subsubsection{Text Styling}
Text styling keys are currently defined from 0x30 - 0x60 for organizational purposes.\\
\begin{tabularx}{\textwidth}{l|l|l|X}
  \textbf{\makecell[cl]{Byte\\Key}} & \textbf{Key Name} & \textbf{Value Format} & \textbf{Description}\\
\hline
30 & Text Color & XX XX XX & RGB value for text color.\\
31 & Highlight Color & XX XX XX & RGB value for text highlight color.\\
35 & Text Alignment & XX & The alignment for the text.\\
3A & Font Name & XX*N 00 & N byte string to set the font name to use. Null terminated.\\
3B & Font Family & XX*N 00 & N byte string to set the font family used. Null terminated.\\
\\
40 & Bold & -- & Set the text to be bold.\\
41 & Italics & -- & Set the text to be italics.\\
42 & Underline & -- & Set the text to be underlined.\\
43 & Strikethrough & -- & Set the text to be strikethrough.\\
44 & Superscript & -- & Set the text to be superscript.\\
45 & Subscript & -- & Set the text to be subscript.\\
\\
50 & \makecell[tl]{Text Scale \\ Window Divider} & XX & This value is only used when the Text Scale Mode (below) is a type of percent. The window size id s divided by this value to give more control of text size. Default value is 4.\\
51 & Text Scale Mode & XX & Determine how the size of the text is calculated. The following are the currently supported values:
\begin{itemize}
\item 01 - Use points to determine the text size.
\item 02 - Scale text size as a percent of the parent element width.
\item 03 - Scale text size as a percent of the parent element height.
\item 04 - Scale text size as a percent of the window width.
\item 05 - Scale text size as a percent of the window height.
\end{itemize}
The default value is 01 for points.\\
52 & Text Size & XX & Decide the value of the text size. Divided by 100 if Text Scale Mode is a percent.\\
& & & \makecell[tl]{The formula for how text size is calculated: \\
If Text Scale Mode is points:\\
Text font size = Text Size value\\
If Text Scale Mode is a percent of the window size:\\
Window Axis Size = Window Width or Window Height based on Text Scale Mode value.\\
Text font size = (Text Size value $/$ 100) $*$ ((Window Axis Size)$/$(Text Scale Window Divider))
}


\end{tabularx}
\section{Content}
\subsection{Content Types}
\subsubsection{External Content Types} % still doing this?
\subsection{Actions in Content}
\subsection{Content Styling}
\subsubsection{Global Styling}
Only an overview of the global style.
\subsubsection{Individual Styling}
Styles are handled individually by the content. These styles are only supported by the following content types:
\begin{itemize}
\item Text
\item Button
\item Bitmap
\end{itemize}
\subsubsection{Styling Order}

\section{Extended Chunk Types}
\subsection{Maintaining Compatibility}
To maintain compatibility with previous IC specification versions, all chunk types in the extended range must include the length of the chunk in bytes after defining its ID COMPONENT. This way, if an unknown chunk type is encountered, it can be skipped.
\subsection{Suggesting a new Chunk Type}
To suggest a new chunk type, follow this process:

\chapter{Extended Chunk Types}
\section{Scrollable Container}

\chapter{Content}
\section{Text}
\subsection{Byte Structure}
UTF-8

\subsection{Control Bytes}

Styles can be applied to text by the following byte sequence: 0xEEBC80, and ended with 0xEEBC81.\\
Actions can be started with 0xEEBC82 and ended with 0xEEBC83.

\begin{tabularx}{\textwidth}{l|l|l|X}
  \textbf{\makecell[cl]{Byte\\Key}} & \textbf{Key Name} & \textbf{Value Format} & \textbf{Description}\\
\hline
80 & Start Style & XX XX & Begin applying the Style with the given ID to the text.\\
81 & End Style & -- & Stop applying the style.\\
82 & Start Action & XX XX & Begin applying a link to the text.\\
83 & End Action & -- & Stop considering the text as a link.\\
\\
90 & \makecell[tl]{Start numbered \\ list} & XX & Begin a numbered list at the given level. Level 1 is the top most level.\\
91 & \makecell[tl]{Start unordered \\ bullet list} & XX & Begin a unordered bullet list at the given level.\\
92 & \makecell[tl]{End bullet \\ list} & -- & End the unordered bullet list.\\
93 & New bullet & -- & Begin a new item in the current list. If it is a numbered list, the number will be incremented by one.\\
94 & Set bullet number & XX & Set the current bullet number for the bulleted list.\\
95 & \makecell[tl]{Set current list \\ level} & XX & Set the current list level.\\
96 & \makecell[tl]{Set list left sub \\ indent} & XX YY YY & For the given XX level, set the left sub indent using the signed YYYY value.\\
97 & \makecell[tl]{Set left sub indent \\ level multiplier} & XX XX & For each list sublevel, the left sub intent is this value multiplied by the level number. Default is 60.\\
\\
A0 & \makecell[tl]{Begin Left Indent \\ and Left Sub Indent} & XX XX YY YY & Begin paragraph left indent with XX in tenths of a millimeter and left subindent with YY in tenths of a millimeter. Bytes are signed.\\
A1 & \makecell[tl]{End Left Indent} & -- & End the current left indent.\\
A2 & \makecell[tl]{Begin Right Indent} & XX XX & Begin paragraph right indent in tenths of millimeter. Bytes are signed.\\
A3 & \makecell[tl]{End Right Indent} & -- & End the current right indent.\\
A4 & \makecell[tl]{Begin Before and \\ After Paragraph \\ Spacing} & XX XX YY YY & Begin before paragraph spacing with XX in tenths of a millimeter and after paragraph spacing with YY in tenths of a millimeter. Bytes are signed.\\
A5 & \makecell[tl]{End Paragraph \\ Spacing} & -- & End the current paragraph spacing.\\
\\
B0 & New Line & -- & Create a new line.\\
B1 & Tabs & ? & Set tab levels.\\
\end{tabularx}

\subsection{Styling}
Bullet lists (numbered and not numbered). Paragraph Indent. Paragraph Spacing. Left indent, right indent. Begin/end paragraph. Begin/end alignment. Line break and new line. Line spacing. Tabs. Sub lists.
\section{Button}
\subsection{Byte Structure}
\subsection{Styling}
\section{Image}
\subsection{Byte Structure}
\subsection{Options}
Key value pairs
\section{Bitmap}
\subsection{Byte Structure}
\subsection{Styling}
For each shape type: line, rectangle, and circle.
\section{Web Page}
\subsection{Byte Structure}
\section{Suggesting a new Content Type}
To suggest a new content type, follow this process:

\end{document}